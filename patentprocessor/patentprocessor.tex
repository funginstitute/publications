%% LyX 2.0.6 created this file.  For more info, see http://www.lyx.org/.
%% Do not edit unless you really know what you are doing.
\documentclass[11pt,twocolumn]{article}
\usepackage[latin9]{inputenc}
\usepackage{listings}
\usepackage{graphicx}

\makeatletter

%%%%%%%%%%%%%%%%%%%%%%%%%%%%%% LyX specific LaTeX commands.
%% Because html converters don't know tabularnewline
\providecommand{\tabularnewline}{\\}

%%%%%%%%%%%%%%%%%%%%%%%%%%%%%% User specified LaTeX commands.

\usepackage{fullpage}\usepackage{morefloats}\usepackage{url}\usepackage{pdftexcmds}\usepackage{listings}
\let\textquotedbl="

\makeatother

\begin{document}

\title{Processing USPTO Patent Data}


\author{Gabe Fierro\\
 Coleman Fung Institute for Engineering Leadership\\
 UC Berkeley\\
 \texttt{fierro@eecs.berkeley.edu}}


\date{\today}

\maketitle
 
\begin{abstract}
We describe a completely automated data process designed to consume
weekly releases of patent grants and applications distributed by the
United States Patent and Trademark Office (USPTO). The process downloads
and unpacks the zipped distribution files, parses the raw data into
a SQL database, and performs various disambiguations and statistical
calculations on the database. 
\end{abstract}

\section{Introduction}

Patent data plays an invaluable role in research into economic trends,
invention, innovation policy and technology strategy. Since the digitization
of patent data starting in 1975, though patent data has been freely
available through the United States Patent and Trademark Office, it
has been difficult to use. We present a substantial improvement in
data quality and accessibility over previous third-party re-releases
of US patent data. This will not only facilitate further research
on up-to-date patent records, but also increase the reproducibility
of previous research results. 
 


\section{Processing Workflow}

\begin{figure*}
\center \includegraphics[width=0.8\textwidth]{figs/dataprocess}
\caption{Full patent data process flow}


\label{fig:dataprocess} 
\end{figure*}


The Fung Institute has developed a robust and fully automated toolchain
for processing and providing high quality patent data intended for
research, as illustrated in Figure~\ref{fig:dataprocess}.

As data is downloaded from the USPTO weekly patent releases, it is
parsed, cleaned and inserted into a SQL database. From this database,
assignee and lawyer disambiguations are performed and the patents
are geocoded with a location-based disambiguation. The output data
from these processes are combined with the historical data from the
Harvard Dataverse Network into a single consolidated database. From
this database, an inventor-level disambiguation can be performed,
and various applications can take advantage of the completed data.
 


\section{Data Sources}

The unified patent dataset is composed of processed data from two
separate sources: the Harvard Dataverse Network (DVN)~\cite{disambiguation}
collection of patent data from 1975 through 2010 and the weekly distributions
of Google-hosted USPTO records~\cite{googlefiles}\cite{googlefiles-applications}.


\subsection{Harvard DVN}

The Harvard DVN patent database consists of data drawn from the National
Bureau of Economic Research (NBER), weekly distributions of patent
data from the USPTO, and to a small extent, the 1998 Micropatent CD
product~\cite{micropatent}. The schema of the database can be found
in the Appendix.

While the Harvard DVN patent database was, prior to the UC Berkeley
patent database, the most extensively complete amalgamation of United
States patent data, it is not without its problems. Firstly, there
is little information as to the actual meanings of the columns in
the databases. Without sufficient prior knowledge of patent structure,
it is difficult to glean the semantic significance of each column.
The names alone are often abbreviated and hard to discern. Secondly,
because the DVN database is a combination of several sources into
a single database schema, certain patent entries from NBER and Micropatent
are incomplete where their data source did not provide all the requisite
data. The data obtained from the weekly distributions suffers from
being made available in several different formats. The parser that
was developed to handle the data is overly complicated and does not
handle edge cases well, resulting in missing patent metadata where
the parser did not account for a subtle change in format~\cite{oldparser}.
This is analyzed in greater detail below.


\subsection{USPTO Weekly Distributions}

\begin{table*}[t]
\center %
\begin{tabular}{|l|l|}
\hline 
Time Span  & Data Format \tabularnewline
\hline 
-1974  & paper-based \tabularnewline
1975  & unknown. Data obtained from Micropatent \tabularnewline
1976 - 2001  & Green Book (CITE) APS key-value \tabularnewline
2001  & SGML ST. 32 v2.4 \tabularnewline
2002 - 2004  & Red Book (CITE) XML ST. 32 v2.5 \tabularnewline
2005  & Red Book XML ST. 36 (ICE) v4.0 \tabularnewline
2006  & Red Book XML ST. 36 (ICE) v4.1 \tabularnewline
2007 - 2012  & Red Book XML ST. 36 (ICE) v4.2 \tabularnewline
2013  & Red Book XML ST. 36 (ICE) v4.3 \tabularnewline
2013 -  & Red Book XML v4.4 \tabularnewline
\hline 
\end{tabular}\caption{Table of USPTO grant data formats}


\label{fig:grantformats} 
\end{table*}
\begin{table*}[t]
\center %
\begin{tabular}{|l|l|}
\hline 
Time Span  & Data Format \tabularnewline
\hline 
-2001 & paper-based \tabularnewline
2001  & XML ST. 32 v1.5\tabularnewline
2002 - 2004  & Red Book (CITE) XML ST. 32 v1.6 \tabularnewline
2005 & Red Book XML ST. 36 (ICE) v4.0\tabularnewline
2006 & Red Book XML ST. 36 (ICE) v4.1 \tabularnewline
2007 - 2012 & Red Book XML ST. 36 (ICE) v4.2 \tabularnewline
2013 -  & Red Book XML ST. 36 (ICE) v4.3\tabularnewline
\hline 
\end{tabular}\caption{Table of USPTO grant data formats}


\label{fig:applicationformats} 
\end{table*}


The USPTO distributions take the form of zip archives containing concatenated
XML (Extensible Markup Language) documents, each of which contains
the full text of each patent grant and patent application issued every
week. Prior to 1975, the USPTO used a purely paper-based system before
transitioning to a raw-text key-value and later SGML-based key-value
store %
\footnote{Standard Generalized Markup Language%
}. Patent documents were made available in the XML format starting
in 2001. Although this data is made freely available, the fact that
digital USPTO patent data spans eight different formats and occupies
more than 70 GB (compressed) over the 37 years of its existence makes
rendering the data into an amenable form a nontrivial problem (see
Table~\ref{fig:grantformats} and Table~\ref{fig:applicationformats}).
Patent application data, though only available in a digital format
back to 2001, is nonetheless available in six different formats~\cite{xmlresources}~\cite{xmlretrospective}.
 


\section{Parsing}

All of the conversion from HTML form to a SQL query is done in \verb`batchsql/models.py`. All of the form variables are given to the TestQuery class through the post variable that we get from django. In post, all the values entered by the user are stored as a dictionary in the form {“field-name”:”value”}. All form elements are broadly categorized into three types: 

\begin{enumerate}
\item Field Variables: These are the columns that the user wants information from. For example, Name of a Patent, or Name of the Inventor of the Patent.
\item Filter Variables: These are the filters specified by the user. They are mostly textboxes or select lists. If a user enters ‘TX’ under the Inventor’s location filter, then all the rows (in the columns specified by field variables) that have the inventor’s location as ‘TX’ will be returned.
\item Miscellaneous: The csrf token, the email address, the file type that the user wants the information in are considered as miscellaneous fields as models.py does not use these fields to make queries.
\end{enumerate}

	In models.py, we have a dictionary which maps the form elements’ names to (table,column) which they represent. For example, the Patent Title represents the patent table and the title column, and hence one of the entries in this dictionary will be \verb`‘pri-title’:(‘patent’, ‘title’)`(where ‘pri-title’ is the name of the field for Patent Title). All fields have a prefix of ‘f’ to separate them from filters. Converting the form elements is a 4 step process:

\begin{enumerate}
\item Get columns that the users want in their results and store it in a set. This is generated from the Field Variables.
\item Get the names of tables to be searched and store it in a set. This is generated from both the Field Variables and Filter Variables.
\item Get the filter conditions and store it in a set. This is generated from both the Field Variables (for cross-referencing between tables) and Filter Variables.
\item Loop through the above sets and construct a query of the structure\\
\end{enumerate}

\begin{center}
SELECT \{table.columns\} FROM \{tables\} WHERE \{filters\};
\end{center}

	Once this query is generated, it is stored in a local databse that stores the queued and completed job information, and then the \verb`run_jobs.py` file gets this query from this database and runs the jobs that have not yet been completed. It uses sqlalchemy$^{[2]}$ to connect  to and execute queries at the remote MySQL$^{[3]}$ database. 


\section{Database}

One of the main purposes of the patent processor project is to provide
a usable database of relevant patent data. This database should facilitate
the retrieval of patent records, citations, inventors, lawyers, assignees,
and other patent-related data. The linked nature of these types of
records suggests that a relational database model would be most suited
to the data, which motivated the decision to model patent data in
SQL. SQL, or Structured Query Language, is a language designed for
managing data held in a relational database.

Because the majority of the data processing pipeline is written in
Python, it is hard to integrate otherwise easy-to-use SQL code. There
are multiple flavors of SQL -- among them, SQLite and MySQL. SQLite
simplifies local development because the whole database is represented
as a single efficiently-sized file that can be copied, moved and manipulated
much like a traditional file. However, it is hampered by a lack of
support for more complex SQL features, and has poor support for concurrent
users (e.g. multiple processes attempting to access the same database).
MySQL offers advanced SQL features (such as \verb`LEFT OUTER JOIN`)
and scales to multiple users and large amounts of data much easier
than SQLite, but requires more specialized knowledge to use and access.
MySQL is more suited for production environments, whereas SQLite is
better for development. We want to be able to easily switch between
these two flavors of SQL depending on our purpose without having to
develop multiple branches of database integration.


\subsection{SQLAlchemy}

\begin{figure*}
\begin{lstlisting}
query = `select * from Patent where \
    number = ``%s''' % patent_number
result = connection.execute(query)
patent_id = result[3]
query = 'select * from assignee \
  where patent_id = ``%s''' % patent_id
connection.execute(query)
\end{lstlisting}
 \label{fig:sql-assignee} \caption{Finding assignees for a patent using traditional Python-SQL}
\end{figure*}
\begin{figure*}
\begin{lstlisting}
patent = session.query(Patent).
    filter_by(number = patent_number)
patent.assignees
\end{lstlisting}
 \label{fig:sa-assignee} \caption{Finding assignees for a patent using SQLAlchemy}
\end{figure*}


SQLAlchemy~\cite{sqlalchemy} is a Object Relational Mapper (ORM)
for Python that seeks to abstract away the differences between SQLite,
MySQL, and other SQL-based relational databases. The SQLAlchemy ORM
maps Python classes to an underlying SQL database such that the database
can be manipulated as though it were a native Python object. This
means that the object model and the database schema can be decoupled,
effectively removing the need for separate lines of development for
each possible database engine.

Database-related code written using SQLAlchemy is much cleaner and
easier to work with than the traditional, kludgy idioms. In the case
of SQLite, the normal Python module requires the user to excute strings
of SQL code: 
\begin{lstlisting}
query = `select * from Patent where \
    number = ``%s''' % patent_number
connection.execute(query)
\end{lstlisting}


Not only does this require the programmer to know SQL syntax, but
this paradigm leaves the database open to SQL injection, wherein unintended
and possibly malicious code is executed on the SQL database. For example,
here, we are operating on the assumption that the variable \verb`patent_number`
contains a valid patent number. It could actually contain the string
\verb`''; delete from Patent;--`, which would terminate the original
\verb`select` statement, delete all entries from the Patent table,
and then exit as though nothing had happened. To avoid such attacks,
it is necessary to sanitize all SQL strings to make sure they contain
valid and safe queries.

SQLAlchemy obviates the need to implement such verbose security methods.
The SQLAlchemy equivalent to the above query is:

\begin{lstlisting}
session.query(Patent).
    filter_by(number = patent_number)
\end{lstlisting}


Immediately, we can see that this code is much simpler and cleaner.
When SQLAlchemy accepts string input, as with the \verb`patent_number`
variable here, it automatically escapes all significant characters
like semicolons and apostrophes, essentially nullifying the possiblity
of SQL injection attacks.

SQLAlchemy further simplifies the handling of foreign keys and complex
joins between tables, and can even implement these features over database
engines (such as SQLite) that do not normally have them. Consider
Figure 3 versus Figure 4.


\subsection{Limitations}

The nice features of SQLAlchemy come at a price. The higher level
interface to the SQL database requires a nontrivial amount of bookkeeping.
Foreign keys lookups and checks introduce a certain amount of overhead,
so when a process loops through a list of database items, multiple
SQL queries can be executed against the backend for each object if
the process asks for linked objects.

SQLAlchemy offers tools to help reduce the number of individual queries
sent to the underlying database, but there is an inescapable overhead
to using an ORM over the raw SQL.


\subsection{New Schema}

\begin{figure*}
\center \includegraphics[width=0.6\textwidth]{figs/database-simplified}
\caption{High level view of new database schema}


\label{fig:newschema} 
\end{figure*}


We wanted to have a highly-linked database that would make it easy
for developers to access related information for a given set of patents.
The DVN schema, as described in the Appendix, does not take advantage
of foreign key relations, and places much manual burden on the user.
This was a primary motivating factor in our design, which is summarized
in Figure~\ref{fig:newschema}.


\subsection{Raw vs Processed}

If we examine the new database schema, for each of the \verb`inventor`,
\verb`lawyer`, \verb`location`, and \verb`assignee` tables, we
can see a ``raw'' version (e.g. \verb`rawinventor`) and a plain
version. The \verb`raw` tables contain the inventor, lawyer, location
and assignee records \emph{as they appear in the USPTO files}, which
means that the naming inconsistencies and misspellings are preserved.
These records are run through disambiguation methods of various degrees
of rigor, and the cleaned records are stored in the plain tables.
See below for a description of these disambiguation methods.

When the cleaned records are inserted, we link them to both the related
patent and the raw version using foreign keys in the database, so
it is simple to examine groups of related records. See Table~\ref{table:rawclean}.

\begin{table*}
\center %
\begin{tabular}{|l|l|l|}
\hline 
Table  & Access  & Value \tabularnewline
\hline 
Patent  & \verb`patent`  & US8434162 \tabularnewline
Inventor  & \verb`patent.inventors[0]`  & Thomas H. Stachler \tabularnewline
Raw Location  & \verb`patent.inventors[0].rawlocation`  & Deyton, OH, US \tabularnewline
Clean Location  & \verb`patent.inventors[0].location`  & Dayton, OH, US \tabularnewline
\hline 
\end{tabular}\caption{Accessing related raw and clean records. Note the spelling correction
in the clean record}


\label{table:rawclean} 
\end{table*}

 


\section{Disambiguations}

One of the primary problems with conducting meaningful research with
USPTO patent data is the high variability in quality. Cities are misspelled
or mislisted. Organizations are alternatively abbreviated and listed
in full with little modicum of consistency. Inventors, lawyers and
assignees will misspell their names, change their names and unpredictably
list their middle initials or names. The Berkeley patent database
provides facilities to account for these errors, and codifies the
disambiguation of such records in order to make possible their accurate
retrieval.


\subsection{Geocoding}

There are over 12 million locations listed in the USPTO patent weekly
downloads from 1975 to 2013, with 350,000 unique tuples of \verb`(city, state, country)`.
These tuples follow the typical motif of data problems in the rest
of the patent data: incorrect or nonstandard country codes, inconsistent
romanization of foreign locations and various misspellings. We resolved
the ambiguities in the location data using a propietary disambiguation
technique developed by Google. When new patent data is processed,
we run a series of data cleaning processes to correct for some of
the common errors, then cross reference with the lookup table~\cite{geotable}
obtained through the Google disambiguation.

A detailed analysis of the problems with USPTO location data and our
handling of locations can be found through a related Fung Institute
publication~\cite{geocoding}.

Locations are associated with assignees, inventors and lawyers. Typically,
a patent record's ``location'' is the location of the first inventor
listed on the patent.


\subsection{Assignees}

For a given patent, the assignees are the entities (either organizations
or individuals) that have property rights to the patent. The assignee
records are imperative for firm-level analysis of patent data, and
are used for tracking ownership of patents. The weekly releases of
patent documents only contain the original assignee of a patent when
it was initially granted.

However, it is difficult to obtain accurate results for simple (and
necessary) questions such as \emph{``which patents are owned by firm
X?''} because of the pandemic inconsistency of spellings. A cursory
search for assignee records that resemble General Electric yields
the following:
\begin{itemize}
\item General Electric Company 
\item General Electric 
\item General Electric Co.. 
\item General Electric Capital Corporation 
\item General Electric Captical Corporation 
\item General Electric Canada 
\item General Electrical Company 
\item General Electro Mechanical Corp 
\item General Electronic Company 
\item General Eletric Company 
\end{itemize}
This is not even a complete list of all the (mis)representations of
General Electric, but already we can see the potential issues with
trying to get accurate results.

We do not yet provide fully featured entity resolution for assignee
records, but we do maintain a preliminary disambiguation of the records
that corrects for minor misspellings. We do this by applying the Jaro-Winkler~\cite{jw}
string similarity algorithm to each pair of raw assignee records.
Two records that are within a certain bound of similarity are considered
the same, and are linked together.

It is not tractable to perform pairwise computation on each of the
5,850,531 raw assignee records in the database (at time of writing),
so we group the assignees by their first letter, and then perform
the pairwise comparisions within each of these blocks. This allows
us to hold a smaller chunk of the assignees in memory at each step,
and achieves near the same accuracy.


\subsection{Lawyers}

The raw lawyer records follow much of the same deficiencies in quality
as the assignee records. Again, we only offer a preliminary disambiguation
of lawyer records using the same algorithm as described above, but
future development will yield more accurate results.


\subsection{Inventors}

We provide a polished disambiguation mechanism for inventor records.
Using the published name of an inventor, the patent technology class,
co-inventor names, published location and original assignee, we are
able to infer with more than 95\% accuracy which inventor records
are the same across all records in the patent database.

A detailed summary of our technique can be found through a related
Fung Institute publication~\cite{newdisambiguation}. 
 


\section{Statistics}

Many research applications of patent data require records from multiple
tables to be linked together: for instance, finding all citations
made to a patent, or finding all patents for an inventor. Due to the
size of the database, however, gathering all the requisite data and
linking it together takes a nontrivial amount of time. To facilitate
some common research vectors, we provide three tables of precompiled
statistics.

The \verb`FutureCitationRank` table contains the rank of each patent
by the number of future citations in each year. This answers the question
``in year X, patent number Y got Z citations. It was the Nth most
cited patent that year''.

The \verb`InventorRank` table contains the rank of each inventor
by how many patents they have been granted in a given year.

The \verb`CitedBy` table contains the direct mapping of a focal patent
to all patents that cite that patent. 


{ {\scriptsize{ \bibliographystyle{acm}
\bibliography{patentprocessor}
 }}}
\end{document}
