%% LyX 2.0.6 created this file.  For more info, see http://www.lyx.org/.
%% Do not edit unless you really know what you are doing.
\documentclass[11pt]{article}
\usepackage[latin9]{inputenc}
\usepackage{url}
\usepackage{graphicx}
\usepackage{minted}
\newminted{shell-session}{mathescape, linenos, numbersep=5pt, frame=lines, framesep=2mm}

\makeatletter
%%%%%%%%%%%%%%%%%%%%%%%%%%%%%% User specified LaTeX commands.

\usepackage{morefloats}\usepackage{fullpage}\usepackage{morefloats}\usepackage{url}\usepackage{pdftexcmds}\usepackage{listings}

\makeatother

\begin{document}

\title{Weekly Updates for Fung Institute Patent Database}


\author{Gabe Fierro\\
 Coleman Fung Institute for Engineering Leadership\\
 UC Berkeley\\
 \texttt{fierro@eecs.berkeley.edu}}


\date{\today}

\maketitle
 


\section{Introduction}

Here, we discuss the infrastructure and configuration for a weekly
process that downloads the latest weekly releases of USPTO patent
files, runs the requisite data transformations, and integrates the
processed data into the larger database.

This system has been fully adjusted to deal with the weekly application
releases as well as the grant releases.


\section{Architecture}

\begin{figure}[h]
\center \includegraphics[width=0.8\textwidth]{figs/dataprocess}
\caption{Toolchain for processing patents}


\label{fig:dataprocess} 
\end{figure}


There are five main stages to the patent processing infrastructure:
\begin{itemize}
\item \textbf{Parse}: Downloads the latest patent files, interprets the
structured data, and inserts the compiled records into the database 
\item \textbf{Clean}: Runs the assignee, lawyer and geographic disambiguations
on the data 
\item \textbf{Consolidation}: Prepares the raw inventor data for being sent
to the disambiguation engine 
\item \textbf{Disambiguation}: Performs entity resolution on the raw inventor
data 
\item \textbf{Integration}: Incorporates the disambiguated inventor records
into the database 
\end{itemize}
The code represents these stages as modular chunks that can be configured
and run independently of each other. For the weekly update system,
these stages need to be run end-to-end with no interspersed configuration.

At the time of writing, the complete process is not fully automated.
The consolidation stage will output a file \verb`disambiguator.csv`
that is formatted for input into the disambiguation engine. After
the disambiguation engine is run, the integration stage must be run
manually. This manual step will be eliminated once the disambiguator
has been properly configured.

% go over the parts of the infrastructure that we're automating
% low memory vs high memory



\section{Configuration}

The `start.py` script provides a simple interface to the patent processor
by means of a configuration file. Here is a sample configuration file:

\begin{verbatim}
[process]
parse=test
clean=True
consolidate=True
doctype=all

[test]
datadir=test/fixtures/xml
grantregex=i?pg\d{6}.xml
applicationregex=i?pa\d{6}.xml

[2012]
downloaddir=tmp
download years=2012
\end{verbatim} 

The \verb`[process]` section is mandatory, as it defines the roadmap
of the data process. Each of the lines below the \verb`[process]`
line are configuration options. \verb`parse` indicates which of the
parsing configurations will be run (see below). \verb`clean`, if
\verb`True`, runs the cleaning step on the output of the previously
run parse. Similarly, \verb`consolidate`, if \verb`True`, runs the
consolidation step on the output of the cleaning step.

The \verb`parse` section takes as input the title of another configuration
section, which defines the options for which patent files are to be
downloaded and parsed. \verb`parse` sections accept the following
configuration options:
\begin{itemize}
\item \textbf{datadir}: specifies the path to the directory containing the
XML files that we wish to parse. The path will be evaluated relative
to the root directory of the preprocessor
\item \textbf{grantregex} and \textbf{dataregex}: specifiy the regular expressions
that matches the filenames of the XML files we want to parse. They
default to \verb`i?pg\d{6}.xml` and \verb`i?pa\d{6}.xml`, which
matches USPTO grant and application files since 2001.
\item \textbf{years}: specifies the range of years we want to download and
parse. If the current year is specified, the script will download
all possible files. If this option is provided, the \verb`datadir`
option will be ignored, and the files will be downloaded to the directory
indicated by the \verb`downloaddir` option (see below). If this option
is \emph{not} provided, then the parser will operate on the contents
of \verb`datadir`.


Years are to be separated by commas, and ranges are indicated by using
dashes. For example, to download the years 1995, 1997, 1998, 1999,
2000, 2001 and 2005, we would indicate this as \verb`years=1995,1997-2001,2005`.

\item \textbf{downloaddir}: specifies the target directory into which the
needed patent files will be downloaded. This directory is evaluated
relative to the root directory of the processor. If it does not exist,
it will be created. If the directory already exists and already contains
pre-downloaded files, the process will only download the needed files
to avoid unnecessary work.
\end{itemize}
The final configuration sections, \verb`[grant-xml-handlers]` and
\verb`[application-xml-handlers]`, specify which XML parser is to
be used for which files. The USPTO has used eight different data schemas
for patent grants and six for applications, and because it is difficult
to automatically detect the schema of an XML document, this section
makes it possible to indicate which XML parser should be used for
which dates of patent releases.

This section should only have to be touched when a new parser is introduced.
A date is indicated by either YYYY or YYYYMMDD, and ranges are indicated
by dashes in between two dates. If only one date is provided, then
the corresponding parser will be used for all subsequent dates. The
current configuration is listed here:

\begin{verbatim}
[grant-xml-handlers]
2005-20130108=lib.handlers.grant_handler_v42
20130115=lib.handlers.grant_handler_v44
default=lib.handlers.grant_handler_v42

[application-xml-handlers]
2001-20060822=lib.handlers.application_handler_v41
20060823-20130116=lib.handlers.application_handler_v42
20130117=lib.handlers.application_handler_v43
default=lib.handlers.application_handler_v42
\end{verbatim} 


\section{Running}

The installation process has been developed and tested on machines
capable of running BASH, which is Mac OS X and most Linux environments.

To setup a local environment to run the preprocessor, it is best practice
to setup \verb`virtualenv` to handle local Python packages.

\begin{shell-sessioncode}
$ git clone  https://github.com/funginstitute/patentprocessor/
$ cd patentprocessor
$ pip install -r requirements.txt
$ wget https://s3.amazonaws.com/fungpatdownloads/geolocation_data.7z
$ mv geolocation_data.7z lib/
$ 7zr x geolocation_data.7z
\end{shell-sessioncode}

Alternatively, use the included \verb`Vagrantfile`, built using the ever-helpful Vagrant virtual machine configuration tool: \url{http://www.vagrantup.com/}.
The Vagrantfile is located in \verb`patentprocessor/vm` and the full development environment can be started simply by running \verb`vagrant up` from the
command line, assuming Vagrant and a VM provider such as Virtualbox (\url{https://www.virtualbox.org/}) are installed. The provisioning for the virtual machine will
automatically configure for running the data process through the consolidation for the disambiguator, so use this as a model for configuring your own environment.

Configure the database connection in \verb`lib/alchemy/config.ini`,
then configure the data process in \verb`process.cfg`. The patent
processor can then be started by running \verb`python start.py process.cfg`.

\subsection{Running Manually}

Instead of running the toolchain through the automated and configurable interface, we can each of the segments discretely. 

\begin{shell-sessioncode}
$ python parse.py -p path/to/data/files
$ ./run_clean.sh
$ ./run_consolidation.sh <optional path to previous inventor disambiguation output>
$ # here we would deliver the 'disambiguator_<Month>_<Day>.tsv' file to the disambiguator
$ python integrate.py <input to disambiguator> <output of disambiguator>
\end{shell-sessioncode}

\section{Running Time}

Current times are for processing granted data only. Times obtained on a 16-core, 2.0 Ghz machine running Fedora 19 with 256 GB of RAM. MySQL server is running on an 8-core, 3.0Ghz machine running Ubuntu 12.04 with 24 GB of RAM.

\begin{center}
\begin{tabular}{| l | c |}
\hline
Task & Time \\ \hline \hline
Parsing 1 week of data & 2 hours \\ \hline
Location Disambiguation & 20 hours \\ \hline
Assignee Disambiguation & 26 hours \\ \hline
Consolidation for Inventor Disambiguation & 24 hours \\ \hline
Inventor Disambiguation & 90 hours \\ \hline
Inventor Ingestion & 8 hours \\ \hline \hline
Total & 170 hours \\ \hline
\end{tabular}
\end{center}

\section{Resources}

\noindent The current production code can be found at

\begin{center}
\url{https://github.com/funginstitute/patentprocessor}
\par\end{center}

\noindent A more detailed description of the patent data process can
be found at

\begin{center}
\url{https://github.com/funginstitute/publications/raw/master/patentprocessor/patentprocessor.pdf}
\par\end{center}
\end{document}
