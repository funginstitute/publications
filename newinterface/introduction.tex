There was a big database which had a lot of information related to patents. This database was divided into tables each of which contained different parts of information (patent information, inventor information, assignee information, lawyer information, etc.). However, before creating this interface, the only way to access this information was by making SQL queries. This required login credentials to the MySQL database, and SQL query forming expertise. To give a person without these privileges access to this information, we needed a web page where users could select what information they want from this database.

	The goal was to make a user friendly web interface which allowed users to select what rows they wanted from the tables in the database, specify any filters they wanted to specify for these rows, enter their email address, and specify the format in which they wanted the information (CSV, TSV, or SQLITE3). To do this, we selected to use the Django~\cite{django} framework because it was easy to install and learn. And then, to run the query itself, we used an external python file that imported the Django settings and just ran jobs if any were available (or sleep for some specific amount of seconds if no jobs were available). Our final goal was to make an app that would do this sort of forms to queries transformation on any database with minimal changes to the files included in the app. 